\documentclass{article}
\usepackage{amsmath}
\usepackage{cancel}
\usepackage{graphicx}

\begin{document}
	
	\title{Actividad 5}
	\author{Juan Rodriguez y Alejandro Castellanos}
	\date{}
	{\Large \maketitle{Problema 1}}
	
	\[
	\int \frac{x-2}{\sqrt{-x^2+2x+1}} \,dx  = a \sqrt{-x^2 + 2x + 1} + k \int \frac{dx}{\sqrt{-x^2 + 2x + 1}}
	\]
	
	\vspace{12pt}
	
	A continuación, se deriva respecto de \( x \):
	
	\[
	\frac{x-2}{\sqrt{-x^2 +2x+1}} = \frac{d}{dx}\left(a\sqrt{-x^2 +2x+1}\right) + \frac{k}{\sqrt{-x^2 +2x+1}}
	\]
	
	\[
	\frac{x-2}{\sqrt{-x^2 +2x+1}} = \frac{a(-2x+2)}{2\sqrt{-x^2+x+1}} + \frac{k}{\sqrt{-x^2 +2x+1}}
	\]
	
	\[
	x-2 = \sqrt{-x^2 +2x+1} \cdot \frac{-2ax+2a+2k}{2\sqrt{-x^2 +2x+1}}
	\]
	
	Simplificamos:
	
	\[
	x-2 = \cancel{\sqrt{-x^2 +2x+1}} \cdot \frac{{-\cancel2ax+\cancel2a+\cancel2k}}{\cancel{2}\cancel{\sqrt{-x^2 +2x+1}}}
	\]
	
	\begin{align*}
		x-2 &= -ax + a + k \\
		x^1 &:\quad 1 = -a \quad \Rightarrow \quad a = -1 \\
		x^0 &:\quad -2 = a + k \quad \Rightarrow \quad k = -1
	\end{align*}
	
	Sustituimos a y k en la primera igualdad:
	
	\[
	-\sqrt{-x^2 + 2x + 1} - \int \frac{1}{\sqrt{-x^2 + 2x + 1}} \,dx = -\sqrt{-x^2 + 2x + 1} - \int \frac{1}{\sqrt{2 - (x-1)^2}} \,dx =
	\]
	
	\[
	-\sqrt{-x^2 + 2x + 1} - \int \frac{\frac{1}{\sqrt{2}}}{1 - \left(\frac{x-1}{\sqrt{2}}\right)^2}\,dx = -\sqrt{-x^2 + 2x + 1} - \int  \frac{\frac{1}{\sqrt{2}}}{1 - \left(\frac{x-1}{\sqrt{2}}\right)^2} \,dx =
	\]	
	
	\[
	\fbox{$-\sqrt{-x^2 + 2x + 1} - \arcsin\left(\frac{x-1}{\sqrt{2}}\right) + C$}
	\]
	
	\section*{}
	{\Large \maketitle{Problema 2}}
	
	\[
	\text{Elipse: } x^2 + 4y^2 = 1
	\]
	
	Volumen al girar al rededor del eje X:
	
	\[
	V_x = \pi \int_{a}^{b} [F(x)]^2 \,dx
	\]
	
	donde \(a\) y \(b\) son los puntos de corte con el eje \(x\).
	
	\vspace{12pt}
	
	Dada la ecuación de la elipse \(x^2 + 4y^2 = 1\), podemos despejar \(y\) como:
	
	\[
	y = \sqrt{\frac{1 - x^2}{4}}
	\]
	
	Por lo tanto, la función \(F(x)\) es:
	
	\[
	F(x) = \sqrt{\frac{1 - x^2}{4}}
	\]
	
	Puntos de corte de \(F(x) = 0\):
	
	\[
	\sqrt{\frac{1 - x^2}{4}} = 0 \quad \Rightarrow \quad x = \pm 1
	\]
	
	Por lo tanto, \(x \in [-1, 1]\). 
	
	\vspace{12pt}
	
	En el primer cuadrante, esto se reduce a \(x \in [0, 1]\).
	
	\vspace{12pt}
	
	La expresión para \(V_x\) es:
	
	\[
	V_x = \pi \int_{0}^{1} \left(\sqrt{\frac{1 - x^2}{4}}\right)^2 \,dx = \frac{\pi}{4} \left[x - \frac{x^3}{3}\right]_{0}^{1} = \frac{\pi}{4} \left(\frac{2}{3} - 0\right) = \boxed{\frac{\pi}{6}}
	\]
	

	
	
	\vspace{12pt}
	
	Volumen al girar al rededor del eje Y:
	
	\[
	V_y = \pi \int_{c}^{d} [F(y)]^2 \,dy
	\]
	
	donde \(c\) y \(d\) son los puntos de corte con el eje \(y\).
	
	\vspace{24pt}
	
	Dada la ecuación de la elipse \(x^2 + 4y^2 = 1\), podemos despejar \(x\) como:
	
	\[
	x = \sqrt{1 - 4y^2} 
	\]
	
	Por lo tanto, la función \(F(y)\) es:
	
	\[
	F(y) = \sqrt{1 - 4y^2} 
	\]
	
	Puntos de corte de \(F(y) = 0\):
	
	\[
	\sqrt{1 - 4y^2} = 0 \quad \Rightarrow \quad y = \pm \frac{1}{2}
	\]
	
	Por lo tanto, \(y \in \left[-\frac{1}{2}, \frac{1}{2}\right]\). 
	
	\vspace{12pt}
	
	En el primer cuadrante, esto se reduce a \(y \in \left[0, \frac{1}{2}\right]\).
	
	La expresión para \(V_y\) es:
	
	\[
	V_y = \pi \int_{0}^{\frac{1}{2}} \left(\sqrt{1 - 4y^2}\right)^2 \,dx = \pi \left[y - \frac{4y^3}{3}\right]_{0}^{\frac{1}{2}} = \pi \left(\frac{1}{3} - 0\right) = \boxed{\frac{\pi}{3}}
	\]
	

	
	\section*{}
	{\Large \maketitle{Problema 3}}
	
	\vspace{12pt}
	
	La longitud de la curva definida por las ecuaciones $x(t) = \frac{1}{3}t^3$ y $y(t) = \frac{1}{2}t^2$ desde $t = 1$ hasta $t = 2$ es:
	
	\[ L = \int_{1}^{2} \sqrt{\left(\frac{dx}{dt}\left(\frac{1}{3}t^3\right)\right)^2 + \left(\frac{dy}{dt}\left(\frac{1}{2}t^2\right)\right)^2} \, dt \]
	
	\vspace{12pt}
	
	Las derivadas de las funciones paramétricas son $x'(t) = t^2$ y $y'(t) = t$.
	
	\vspace{12pt}
	
	\[ L = \int_{1}^{2} \sqrt{t^4 + t^2} \, dt = \int_{1}^{2} t \sqrt{t^2 + 1} \, dt = \frac{1}{2} \int_{1}^{2} 2t\sqrt{t^2 + 1} \, dt = \]
	
	\vspace{12pt}
	
	\[ = \frac{1}{2} \left[ \frac{2}{3} (t^2 + 1)^{3/2} \right]_{1}^{2} = \frac{1}{3} \left[ \sqrt{(2^2 + 1)^3} - \sqrt{(1^2 + 1)^3} \right] = \frac{1}{3} \left[ \sqrt{125} - \sqrt{8} \right] = \]
	
	\vspace{12pt}
	
	\[ = \frac{1}{3} \left[ (5 \sqrt{5} - 2 \sqrt{2}) \right] \approx \boxed{2.784} \]
	

	
\end{document}