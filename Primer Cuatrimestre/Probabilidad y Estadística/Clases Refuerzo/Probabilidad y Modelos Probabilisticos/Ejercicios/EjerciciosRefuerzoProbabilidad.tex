\documentclass[fleqn]{article}
\usepackage{amsmath}
\usepackage{amsfonts}

\title{Ejercicios Probabilidad y Modelos Probabilísticos}
\author{Juan Rodríguez}

\begin{document}
	\maketitle
	\section{Ejercicio 1 - Parcial MAIS 2023}
	En una población, operan dos compañías de telefonía móvil, A y B, de forma que un 60\% de los usuarios tienen contrato con A y el 40\% lo hacen con la compañía B. \\
	Los vecinos de esa población pueden utilizar sólo dos marcas de teléfono, T1 y T2; el 70\% de ellos han elegido la marca T1. Además un 30\% disponen de teléfonos de ambas marcas y todos los vecinos tienen teléfono. \\
	La probabilidad de que se produzca un corte en la comunicación durante una llamada es de 0,1 para los usuarios de A, de 0,15 para los usuarios de la compañía B y de 0,05 para quienes utilizan un terminal T1.
	\begin{enumerate}
		\item[a)] Un individuo de la población está haciendo una llamada. Calcular la probabilidad de que se corte la comunicación.
		\item[b)] ¿Cuál es la probabilidad de que un vecino tenga sólo un teléfono de la marca T2?
		\item[c)] A un usuario se le ha cortado la comunicación. ¿Cuál es la probabilidad de que su teléfono sea un T1?
		\item[d)] Un usuario no tiene terminal de marca T1. ¿Cuál es la probabilidad de que se le corte la comunicación?
	\end{enumerate}
	\section{Ejercicio 2 - Final MAIS 2022 Modificado}
	En una compañía de seguros se recogen comunicaciones de incidencias de los clientes que
	cuentan con un seguro de vivienda. Las incidencias se clasifican en dos categorías: D
	(daños en el hogar) y R (robo). \\
	Durante el último mes se han recibido un total de 86 comunicaciones de las que 24
	incluyen incidencias de tipo R y 68 incluyen incidencias de tipo D. No hay incidencias que
	no sean robo o daños en el hogar. \\
	Una vez analizadas todas las comunicaciones, la compañía decide pagar compensación a
	un 65\% de las reclamaciones por robo y a un 80\% de las reclamaciones por daños en el
	hogar.
	\begin{enumerate}
		\item[a)] ¿Qué porcentaje de reclamaciones incluyen incidencias de los dos tipos?
		\item[b)] Indicar razonadamente si los sucesos R y D son independientes
		\item[c)] ¿Cuál es la probabilidad de que una reclamación sea por robo y no se acepte la compensación?
		\item[d)] ¿y cuál es la probabilidad de que sea por daños y con compensación?
	\end{enumerate}
	\section{Ejercicio 3}
	El 1\% de la población de un determinado lugar padece una enfermedad. Para detectar esta enfermedad se
	realiza una prueba diagnóstica. Esta prueba da positiva en el 97\% de los pacientes que padecen la enfermedad;
	en el 98\% de los individuos que no la padecen da negativa. Si elegimos al azar un individuo de esa población:
	\begin{enumerate}
		\item[a)]¿Cuál es la probabilidad de que un individuo dé positivo y padezca la enfermedad?
		\item[b)]Si sabemos que ha dado positiva, ¿cuál es la probabilidad de que padezca la enfermedad?
	\end{enumerate}
	\section{Ejercicio 4}
	La probabilidad de que el alumno que está leyendo este examen no haya asistido a clase durante el curso es
	de 2/3. Sin embargo, el alumno tiene que hacer este examen si quiere aprobar el curso. Este examen es difícil
	y la probabilidad de aprobarlo habiendo ido a clase es la misma que la de suspenderlo habiendo ido a clase.
	Pero tiene solamente un 0.25 de probabilidad de aprobarlo si no ha asistido antes a clase durante el curso.
	Cuando el profesor corrige el examen, se encuentra con que el alumno ha suspendido, ¿cuál es la probabilidad
	de que el alumno haya asistido a clase?
	\section{Ejercicio 5}
	En una región, el número medio de empresas con más de 100 trabajadores que presentaron suspensión de pagos ha sido de 6,8 por año. Obtener:
	\begin{enumerate}
		\item[a)] probabilidad de que ninguna empresa de más de 100 trabajadores presente suspensión de pagos durante un trimestre
		\item[b)] probabilidad de que por lo menos dos empresas de más de 100 trabajadores presenten suspensión de pagos durante un año determinado.
	\end{enumerate}
	\section{Ejercicio 6}
	En un gran campo se distribuyen al azar las langostas de acuerdo con una distribución de Poisson de parámetro 2 por kilómetro cuadrado. ¿Cómo tendrá que ser el radio de una región para que la probabilidad de encontrar allí al menos una langosta sea de 0,99?
	\section{Ejercicio 7}
	La dureza Rockwell de una aleación de metales está normalmente distribuida con una media de 70 unidades y una desviación típica de 3
	\begin{enumerate}
		\item[a)] Si una probeta se considera aceptable sólo cuando su dureza está comprendida entre 67 y 75, ¿cuál es el porcentaje de probetas que se rechazan?
		\item[b)] ¿Cuál es la probabilidad de que como mucho 8 de 10 probetas independientemente seleccionadas tengan una dureza inferior a 73,84?
		\item[c)] ¿y cuál es la probabilidad de que esa misma situación se de en un máximo de 80 de un total de 100 probetas?
	\end{enumerate}
	\section{Ejercicio 8 - Parcial MAIS 2023}
	Un sistema informático tiene un componente cuyo tiempo de operación en años, antes de fallar, es una variable $T$ que sigue una distribución exponencial de tiempo medio 5 años.
	\begin{enumerate}
		\item[a)] Se han instalado 7 de estos componentes en diferentes sistemas. ¿Cuál es la probabilidad de que al cabo de 8 años sigan funcionando al menos dos de ellos?
		\item[b)] Si una empresa informática adquiere para sus sistemas 4 lotes de 20 componentes, ¿cuál es la probabilidad de que haya al menos un lote en el que ninguna de sus componentes siga funcionando después de 8 años?
		\item[c)] Calcular $F(10)$ siendo $F(t)$ la función de distribución de $T$.
		\item[d)] ¿Cuántos componentes tendremos que probar de media hasta encontrar el tercero que sigue funcionando tras 8 años?
	\end{enumerate}
	
\end{document}